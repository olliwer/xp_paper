% This is lnbip.tex the demonstration file of the LaTeX macro package for
% Lecture Notes in Business Information Processing from Springer-Verlag.
% It serves as a template for authors as well.
% version 1.0 for LaTeX2e
%
\documentclass[lnbip]{svmultln}
%
\usepackage{makeidx}  % allows for indexgeneration
% \makeindex          % be prepared for an author index
%
\begin{document}
%
\mainmatter              % start of the contribution
%
\title{Title here}
%
\titlerunning{Abbreviated title asd}  % abbreviated title (for running head)
%                                     also used for the TOC unless
%                                     \toctitle is used
%
\author{Olli Rissanen\inst{1} %\and Roger Temam\inst{2}
%Jeffrey Dean \and David Grove \and Craig Chambers \and Kim~B.~Bruce \and
}
%
\authorrunning{Olli Rissanen}   % abbreviated author list (for running head)
%
%%%% list of authors for the TOC (use if author list has to be modified)
\tocauthor{Olli Rissanen}
%
\institute{University of Helsinki, Helsinki, Finland,\\
\email{olli.rissanen@aalto.fi},\\ WWW home page:
\texttt{http://users/\homedir iekeland/web/welcome.html}
}

\maketitle              % typeset the title of the contribution
% \index{Ekeland, Ivar} % entries for the author index
% \index{Temam, Roger}  % of the whole volume
% \index{Dean, Jeffrey}

\begin{abstract}        % give a summary of your paper
Delivering more value to the customer is the goal of every software company. In modern software business, delivering value in real-time requires a company to utilize real-time deployment of software, data-driven decisions and empirical evaluation of new products and features. These practices shorten the feedback loop and allow for faster reaction times, ensuring the development is focused on features providing real value. This thesis investigates practices known as continuous delivery and continuous experimentation as means of providing value for the customers in real-time. Continuous delivery is a development practice where the software functionality is deployed continuously to customer environment. This process includes automated builds, automated testing and automated deployment. Continuous experimentation is a development practice where the entire R\&D process is guided by conducting experiments and collecting feedback. As a part of this thesis, a case study is conducted in a medium-sized software company. The research objective is to analyze the challenges, benefits and organizational aspects of continuous delivery and continuous experimentation in the B2B domain. The data is collected from interviews conducted on members of two teams developing two different software products. The results suggest that technical challenges are only one part of the challenges a company encounters in this transition. For continuous delivery, the company must also address challenges related to the customer and procedures. The core challenges are caused by having multiple customers with diverse environments and unique properties, whose business depends on the software product. Some customers also require to perform manual acceptance testing, which slows down production deployments. For continuous experimentation, the company also has to address challenges related to the customer and organizational culture. An experiment which reveals value for a single customer might not reveal as much value for other customers due to unique properties in each customers business. Additionally, the speed by which experiments can be conducted is relative to the speed by which production deployments can be made. The benefits found from these practices support the case company in solving many of its business problems. The company can expose the software functionality to the customers from an earlier stage, and guide the product development by utilizing feedback and data instead of opinions.
%                         please supply keywords within your abstract
\keywords {Continuous delivery, Development process, B2B}
\end{abstract}
%
\section{Introduction}

\subsection{Problem}
\subsection{Background}
\subsection{Structure of the paper}
%

\section{Related works}

\section{Case study}

\subsection{Objective}
\subsection{Research method}
\subsection{Execution}

\section{Results}

\subsection{RQ1}
\subsection{RQ2}

\section{Discussion}
\section{Summary}
\subsection{Future research}

\paragraph{Notes and Comments.}
The first results on subharmonics were
obtained by Foster and Kesselman in \cite{fos:kes}, who showed the existence of
infinitely many subharmonics both in the subquadratic and superquadratic
case, with suitable growth conditions on $H'$. Again the duality
approach enabled Foster and Waterman in \cite{fos:kes:2} to treat the
same problem in the convex-subquadratic case, with growth conditions on
$H$ only.

Recently, Smith and Waterman (see \cite{smit:wat} and May et al. \cite{mes})
have obtained lower bound on the number of subharmonics of period $kT$,
based on symmetry considerations and on pinching estimates, as in
Sect.~5.2 of this article.

%
% ---- Bibliography ----
%
\begin{thebibliography}{5}

\bibitem{smit:wat} Smith, T.F., Waterman, M.S.: Identification of Common Molecular
Subsequences. J. Mol. Biol. 147, 195--197 (1981)

\bibitem{mes} May, P., Ehrlich, H.C., Steinke, T.: ZIB Structure Prediction Pipeline:
Composing a Complex Biological Workflow through Web Services. In: Nagel,
W.E., Walter, W.V., Lehner, W. (eds.) Euro-Par 2006. LNCS, vol. 4128,
pp. 1148--1158. Springer, Heidelberg (2006)

\bibitem{fos:kes} Foster, I., Kesselman, C.: The Grid: Blueprint for a New Computing
Infrastructure. Morgan Kaufmann, San Francisco (1999)

\bibitem{cff} Czajkowski, K., Fitzgerald, S., Foster, I., Kesselman, C.: Grid
Information Services for Distributed Resource Sharing. In: 10th IEEE
International Symposium on High Performance Distributed Computing, pp.
181--184. IEEE Press, New York (2001)

\bibitem{fos:kes:2} Foster, I., Kesselman, C., Nick, J., Tuecke, S.: The Physiology of the
Grid: an Open Grid Services Architecture for Distributed Systems
Integration. Technical report, Global Grid Forum (2002)

\bibitem{url} National Center for Biotechnology Information, http://www.ncbi.nlm.nih.gov

\end{thebibliography}
%
\end{document}
